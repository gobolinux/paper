%% LyX 2.1.0 created this file.  For more info, see http://www.lyx.org/.
%% Do not edit unless you really know what you are doing.
\documentclass[english]{article}
\usepackage[T1]{fontenc}
\usepackage[utf8]{inputenc}
\usepackage[a4paper]{geometry}
\geometry{verbose,tmargin=3cm,bmargin=3cm,lmargin=3cm,rmargin=3cm}

\makeatletter
%%%%%%%%%%%%%%%%%%%%%%%%%%%%%% User specified LaTeX commands.
\usepackage{url}

\makeatother

\usepackage{babel}
\begin{document}

\title{Understanding package management by analogy to programming languages}

\maketitle

\section{Introduction}

Package management is an area that lies somewhere in the border between
programming languages, operating systems and system administration.
For this reason, it seems to be overlooked by all three fields as
an implementation issue. In the meantime, package management keeps
growing in complexity. New languages, new deployment models and new
portability requirements, all give rise to new package management
systems. Further, this is not simply a matter of competing implementations:
modern complex environments often require several package managers
to be used in tandem.

For example, when writing JavaScript web applications on a Mac environment,
a developer may require using Bower \cite{bower}, a package manager
for client-side JavaScript components. Bower is installed using \texttt{npm}
\cite{npm}, a package manager for \texttt{node.js} \cite{nodejs},
a JavaScript environment. On a Mac system, the typical way to install
command-line tools such as \texttt{npm} is via either Homebrew \cite{homebrew}
or MacPorts \cite{macports}, the two most popular general-purpose
package managers for Mac OSX. This is not a deliberately contrived
example; it is the regular way to install development modules for
a popular language in a modern platform.

In \cite{Burns2012Deploying} we have another example of a typical
software stack, where a deployment and management scenario for Ruby
on Rails applications is described combining a number of tools. It
uses Vagrant \cite{Hashimoto2013Vagrant} for virtual machine management;
Puppet \cite{Arrundel2011Puppet} for editing system configuration
files and driving the system-wide package manager on servers; Capistrano
for deploying the Ruby on Rails application, including installing
Ruby scripts and migrating database tables, driving RubyGems, the
language-specific package manager for Ruby modules (with Bundler to
mitigate module version conflicts); and RVM for managing conflicting
versions of Ruby itself. It is interesting to note the number of different
tools being used on top of each other to manage containment and compatibility
issues on various layers; and again, this is a typical, realistic
scenario.

The combinations of package managers change as we move to a different
operating system or use a different language. Learning one's way through
a new language or system, nowadays, includes learning one or more
packaging environments. As a developer of modules, this includes not
only using package managers but also learning to deploy code using
them, which includes syntaxes for package specification formats, dependency
and versioning rules and deployment conventions. Simply ignoring these
environments and managing modules and dependencies by hand is tempting,
but the complexity of heterogenous environments and keeping track
of dependency updates can become burdensome — all these package managers
were created to solve pratical problems which the developer would
have to otherwise directly handle, after all. Another alternative
that is often proposed, especially by users of operating systems that
feature a system-provided package manager (as is the case of most
Linux distributions), is to avoid using multiple package managers
and use a single general-purpose package manager. This is, of course,
as much as a solution as trying to make everyone agree on a single
programming language, and this is the first of various analogies between
package management and programming languages that we will make throughout
this paper. The result is that the ecosystem is not getting any simpler,
and at first glance it seems that package management is indeed a largely
unsolved problem.

However, maybe the statement ``package management is an unsolved
problem'' simply does not make sense, and is akin to saying that
``programming languages are an unsolved problem''. In the programming
languages world we accept that the multitude of languages is a given.
Beyond that, we understand that there are families of languages with
different paradigms, with well-known tradeoffs. We also accept that
there is room for domain-specific languages and for general-purpose
languages. Most importantly, we know how to set boundaries for each
language and how to make DSLs and general-purpose languages interact.

We argue that all of these observations can be made with regard to
package management as well. In this paper, we will discuss how these
observations map from the world of programming languages to that of
package managers. Most existing package management systems, however,
are still oblivious to the fact that they exist as part of a larger
ecosystem, with parts of it handled by other package managers. By
discussing how programming languages deal with these issues, we point
to directions on how package manegers could follow their example,
drawing on our experiences developing both a system-wide package manager
\cite{Muhammad2002WSL,Homer2010LCA} and a language-specific package
manager \cite{Muhammad2013LuaRocks}. 


\section*{Paradigms of package management}
\begin{quote}
\emph{An analogy between programming language paradigms and package
management paradigms}
\end{quote}
We went through filesystem vs. database in Gobo, but this appears
everywhere.
\begin{itemize}
\item filesystem paradigm vs. database paradigm
\item trade-offs: how runtime lookup of files happens

\begin{itemize}
\item databases involved in runtime lookup: gtk icon cache, Haskell db updates,
etc.
\item pkg-config, how does it fit?
\item filesystem paradigm and runtime lookup: environment variables, index
directories with symlinks, symlinks in lib (libfoo.so, libfoo.so.1,
etc.)
\end{itemize}
\item versioning in db vs filesystem. libfoo.so vs. libfoo.so.1, versioning
in gobo, /usr/include/python2.7/ etc.
\item changes in filesystems to better support the filesystem paradigm?
(union mounts, etc)
\item related work. outside of ``classic'' Unix: appdirs, Windows, Plan
9.
\item filesystem paradigm in ``classic'' Unix: /opt

\begin{itemize}
\item parts of the FS hierarchy which use the filesystem paradigm: /usr/share/appname
\item parts which don't: /usr/share/icons

\begin{itemize}
\item effect this had on share/ in gobo
\end{itemize}
\end{itemize}
\item Makefiles with install and uninstall rules in ports: early example
of database paradigm?
\item other instances of filesystem vs database paradigm:

\begin{itemize}
\item sysv and BSD init versus systemd
\item Mac OSX equivalent
\item /etc text-based config files vs. gconf
\item Windows registry.
\end{itemize}
\end{itemize}

\section*{Language-oriented vs. language-agnostic package managers}
\begin{quote}
\emph{An analogy between PL space with DSLs and general purpose languages
and pkgmgr space with language-oriented and language-agnostic managers}\end{quote}
\begin{itemize}
\item My paper on LuaRocks makes a case for language-oriented package managers:

\begin{itemize}
\item scalability (30,000 Ubuntu packages in total, 50,000+ Ruby gems alone)
\item portability (language managers are used in several OSes)
\end{itemize}
\end{itemize}

\section*{Integration between languages vs. integration between package managers}
\begin{quote}
\emph{An analogy between PL interaction (APIs and FFIs) and pkgmgr
interaction (virtually none??)}
\end{quote}
In GoboLinux we researched the idea of building an ``FFI'' of sorts
into our general-purpose package manager, called Aliens {[}cite Michael's
presentation in linux.conf.au{]}. The idea was that we could provide
a general API for writing shims that interacted with domain-specific
package managers (that is, language-specific ones) in an clean way.
Ad-hoc integrations between general-purpose and domain-specific package
managers proved traumatic to some projects {[}debian's troubles with
rubygems{]}. Still, they have been successfully realized in others,
such as the use of LuaRocks by Buildroot.

language-oriented (domain-specific) vs. language-agnostic (general-purpose)
package managers


\section*{Practical experience with filesystem-oriented package management:
Gobo}
\begin{itemize}
\item earliest related work: GNU stow, encap. (Our paper from Workshop em
Software Livre mentions that, but we didn't really know about them
when making Gobo)
\item GoboLinux

\begin{itemize}
\item Our extensive use of shell scripting for system management was a sort
of attempt to show that the system could work with ``just a few scripts''.
\item core ideas work (the system is still driven via symlinks) but management
scripts became complex over time
\end{itemize}
\item the build system, the tooling for generating packages, is the central
piece of a package management system.

\begin{itemize}
\item not clear at first, but looking at package management systems they
all integrate the build process. Why?
\end{itemize}
\item Discuss the challenges of GoboLinux through the prism of its build
system (?)

\begin{itemize}
\item Tried to make it easy for users to build packages.
\item Took inspiration from systems such as Gentoo, which was itself inspired
from BSD Ports.
\item aim was to make the simple cases super-simple (like a 3-line script)
and the complex cases possible (leveraging the generality of shell
scripts)
\item This worked up to a point. Eventually, started requiring more and
more metadata, even for the so-called simple cases.
\item Further, this metadata had to be integrated with the deployed system. 
\item The ArchLinux build system seems to be a modern-day successor of this
style of build system in the Linux space.
\end{itemize}
\item Modern examples

\begin{itemize}
\item The Nix project has been around almost as long as GoboLinux. NixOS
is nowadays a serious contender in the world of server-oriented operating
systems.
\item virtual machines: minimalism is making its way back in OS layout design.
(There have always been minimalistic Linux distributions, back from
the ``rescue'' distros such as Damn Small Linux and tomsrtbt (which
would fit in a floppy!).

\begin{itemize}
\item CoreOS, based on Gentoo, is the current representative of the minimalistic
server-oriented distro world.
\item Ubuntu Core was recently announced, and this might be a beginning
of a general trend of ``core'' OSes.

\begin{itemize}
\item Ubuntu Core ``snappy'' packages strongly resemble GoboLinux!
\end{itemize}
\end{itemize}
\item Homebrew, a package manager for Mac OS X, is a successful realization
of this idea. One of its original design criteria was to do package
management ``the GoboLinux way'' {[}in the git history of homebrew's
README.md we find them citing gobolinux{]} (so I guess that's the
most widespread legacy of our work)
\end{itemize}
\item What do we mean by filesystem facilities: a more powerful fs would
make things better?

\begin{itemize}
\item Back in 200x I wrote {[}the ``clueless'' whitepaper{]} how we needed
more low-level tooling from the underlying operating system in order
to be able to realize some of the ideas of GoboLinux cleanly. We were
asking for more abstraction and isolation in userspace: essentially
I wanted union filesystems and possibly some sort of containers. Lacking
those, we had to make do with chroot.
\item For a while we hoped that as underlying technology matured, these
ideas could come to fruition.
\item Docker seems to be proof of that; a container-based system that greatly
simplified application deployment.
\item However, Glauber Costa, one of the developers of the Linux Containers
system, described the limitations of that approach exposed the hackery
involved {[}did he write about it somewhere?{]}. Costa himself moved
away from containers and joined the efforts of the OSv project, a
minimalistic operating system targeting hypervisor-based architectures.
\end{itemize}

\bibliographystyle{plain}
\bibliography{gobo}
\end{itemize}

\end{document}
